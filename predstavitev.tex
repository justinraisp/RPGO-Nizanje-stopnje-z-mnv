\documentclass{beamer}

% Nastavitve stila predstavitve
\usetheme{Madrid}
\usecolortheme{beaver}

% Naslovna stran
\title[Nizanje stopnje Bézierjevih krivulj]{Nizanje stopnje Bézierjevih krivulj z metodo najmanjših kvadratov}
\author{Luka Polanič, Justin Raišp}
\date{Ljubljana, 2024}

\begin{document}

% Naslovni diapozitiv
\begin{frame}
    \titlepage
\end{frame}

% Uvod
\begin{frame}{Uvod}
    \begin{itemize}
        \item Obdelujemo problem zniževanja stopnje Bézierjevih krivulj iz $n$ na $m$, kjer $m < n$.
        \item Cilj je minimizacija $L_2$-norme med originalno in aproksimirano krivuljo.
        \item V nalogi uporabimo znanje o višanju stopnje Bézierjevih krivulj za razvoj postopka zniževanja.
    \end{itemize}
\end{frame}

% Motivacija
\begin{frame}{Motivacija}
    \begin{itemize}
        \item Bézierjeve krivulje se pogosto uporabljajo v računalniški grafiki in CAD sistemih.
        \item Nižja stopnja krivulje omogoča:
        \begin{itemize}
            \item Optimizacijo shranjevanja podatkov.
            \item Učinkovitejšo obdelavo in izris.
        \end{itemize}
        \item Problem formuliramo kot minimizacijo $L_2$-norme:
        \[ d_2(p_n, q_m) = \sqrt{\int_0^1 \|p_n(t) - q_m(t)\|^2 dt}. \]
    \end{itemize}
\end{frame}

% Definicije
\begin{frame}{Definicije}
    \begin{itemize}
        \item Bézierjeva krivulja stopnje $n$:
        \[ p_n(t) = \sum_{i=0}^n b_i B_i^n(t), \quad t \in [0, 1], \]
        kjer so $b_i$ kontrolne točke, $B_i^n(t) = \binom{n}{i} t^i (1-t)^{n-i}$ pa Bernsteinovi bazni polinomi.
        \item Bézierjeva krivulja stopnje $m < n$:
        \[ q_m(t) = \sum_{i=0}^m c_i B_i^m(t), \quad t \in [0, 1]. \]
        \item Cilj: Določiti $c_i$, da je $L_2$-razdalja minimalna.
    \end{itemize}
\end{frame}

% Osnovni izrek
\begin{frame}{Osnovni izrek}
    \begin{block}{Izrek 3.2}
        Naj bo $p_n$ Bézierjeva krivulja stopnje $n$ z $\Delta^n b_0 \neq 0$ in $2\alpha \leq n$. 
        Faktorji $\lambda_i$ so podani kot:
        \[ \lambda_i = \left(\frac{2n}{n + 2\alpha}\right)^{-1} \sum_{j=0}^i \binom{n}{j - \alpha} \binom{n}{j + \alpha}, \]
        \noindent kontrolne točke $c_i$ pa izračunamo z:
        \[ c_i = (1 - \lambda_i)c_i^{(I)} + \lambda_i c_i^{(II)}. \]
        Tukaj $c_i^{(I)}$ in $c_i^{(II)}$ določimo z enačbama (5) in (6).
    \end{block}
\end{frame}

% Algoritem
\begin{frame}{Algoritem}
    \begin{itemize}
        \item Začetni podatki: kontrolne točke $b_0, \ldots, b_n$ in ciljna stopnja $m$.
        \item Postopek:
        \begin{enumerate}
            \item Izračun $c_i^{(I)}$ in $c_i^{(II)}$ za stopnjo $n$ do $n-1$.
            \item Izračun kombinacije $c_i$ z utežmi $\lambda_i$.
            \item Ponavljanje postopka za nižanje stopnje do $m$.
        \end{enumerate}
        \item Lastnosti algoritma:
        \begin{itemize}
            \item Ohranja zveznost do reda $\alpha-1$ v robnih točkah.
            \item $L_2$-norma ostaja minimalna.
        \end{itemize}
    \end{itemize}
\end{frame}

\end{document}
