\documentclass{beamer}
\usepackage{array}
\usepackage{amsmath}
\usepackage{booktabs}

% Nastavitve stila predstavitve
\usetheme{Madrid}
\usecolortheme{beaver}

% Naslovna stran
\title[Nižanje stopnje Bézierjevih krivulj]{Nižanje stopnje Bézierjevih krivulj z metodo najmanjših kvadratov}
\author{Luka Polanič, Justin Raišp}
\date{Ljubljana, 2024}

\begin{document}

% Naslovni diapozitiv
\begin{frame}
    \titlepage
\end{frame}

% Motivacija
\begin{frame}{Motivacija}
    Podano imamo Bézierjevo krivuljo stopnje $n$:
    \[ p_n(t) = \sum_{i=0}^n b_i B_i^n(t), \quad t \in [0, 1], \]
    kjer so $b_i$ kontrolne točke in $B_i^n(t)$ Bernsteinovi bazni polinomi stopnje $n$.
    Naš cilj je poiskati Bézierjevo krivuljo stopnje $m < n$
    \[ \tilde{p}_m(t) = \sum_{i=0}^m c_i B_i^m(t), \quad t \in [0, 1], \]
     da je razdalja med krivuljama minimalna.
\end{frame}

% Definicije
\begin{frame}{Metoda najmanjših kvadratov}
  Kontrolne točke $c_i$ določimo tako, da minimiziramo $L_2$-normo med krivuljama $p_n$ in $\tilde{p}_m$, pri čemer je $L_2$-norma definirana kot:
  \[ d_2(p_n, \tilde{p}_m) = \sqrt{\int_0^1 \|p_n(t) - \tilde{p}_m(t)\|^2 \, dt}, \]
  kjer je $\|p_n(t) - \tilde{p}_m(t)\|^2$ kvadrat evklidske razdalje med krivuljama.
\end{frame}

% Osnovni izrek
\begin{frame}{Konstrukcija nižanja stopnje iz $n$ na $n-1$.}
Konstrukcije se bomo lotili induktivno. Če bi želeli zvišati stopnjo Bézierjeve krivulje, bi uporabili naslednjo zvezo: 
\[
b_i = \frac{i}{n} c_{i-1} + \frac{n-i}{n} c_i, \quad i = 0, 1, \ldots, n.
\]
Sedaj lahko na dva načina izrazimo zaporedje neznanih kontrolnih točk $\{c_i\}_{i=0}^{n-1}$.
Dobimo dva sistema enačb, pri čemer bomo v obeh primerih zanemarili eno enačbo,
saj bi bil sistem sicer predoločen.
\end{frame}

\begin{frame}
  \begin{itemize}
    \item Zanemarimo zadnjo enačbo za $i = n$ in dobimo
    \[
    c_i^{(I)} = \frac{1}{n - i} \left( n b_i - i c_{i-1}^{(I)} \right) \quad \text{za} \quad i = 0, 1, \ldots, n - 1.
    \]
    Dodatno upoštevamo, da je $c_{-1} = 0$.
    \item Pri drugi izražavi zanemarimo prvo enačbo, tj. za $i = 0$, in dobimo
    \[
    c_{i-1}^{(II)} = \frac{1}{i} \left( n b_i - (n - i) c_i^{(II)} \right) \quad \text{za} \quad i = n, \ldots, 1.
    \]
  \end{itemize}
  Množici kontrolnih točk $\{c_i^{(I)}\}_{i=0}^{n-1}$ in $\{c_i^{(II)}\}_{i=0}^{n-1}$ predstavljata kontrolna poligona za dve različni Bézierjevi krivulji stopnje $n - 1$. Označimo ju z $\tilde{p}_{n-1}^{(I)}$ in $\tilde{p}_{n-1}^{(II)}$.
\end{frame}


\begin{frame}{Lastnosti krivulj \(\tilde{p}_{n-1}^{(I)}\) in \(\tilde{p}_{n-1}^{(II)}\)}
    Krivulji \(\tilde{p}_{n-1}^{(I)}\) in \(\tilde{p}_{n-1}^{(II)}\) imata naslednji lastnosti:
    \begin{itemize}
        \item \(p_n\) in \(\tilde{p}_{n-1}^{(I)}\) imata enaki vrednosti v \(t = 0\),
        \item \(p_n\) in \(\tilde{p}_{n-1}^{(II)}\) imata enaki vrednosti v \(t = 1\).
    \end{itemize}
\end{frame}


\begin{frame}
  Sedaj moramo poiskati še kontrolne točke $\{c_i\}_{i=0}^{n-1}$ končne Bézierjeve krivulje $\tilde{p}_{n-1}$.
  Za kontrolne točke vzamemo linearno kombinacijo točk, določenih z zgornjima izrazoma:
  \[ c_i = (1 - \lambda_i) \cdot c_i^{(I)} + \lambda_i \cdot c_i^{(II)} \quad \text{za} \quad i = 0, 1, \ldots, n - 1. \]
  Z uvedbo faktorjev $\{\lambda_i \in \mathbb{R}\}$ prevedemo problem iskanja kontrolnih točk $\{c_i\}_i$ na problem računanja ustreznih faktorjev $\{\lambda_i\}_i$, za katere velja, da je $d_2(p_n, \tilde{p}_{n-1})$ minimalna.
\end{frame}

\begin{frame}{Izbira uteži}
   Da se pokazati, da če za krivuljo $p_n$ z $\Delta_n b_0 \neq 0$ in $2\alpha \leq n$, izberemo faktorje $\lambda_i$ kot: 
\[
\lambda_i = \binom{2n}{n + 2\alpha}^{-1} \cdot \sum_{j=0}^i \binom{n}{j - \alpha} \binom{n}{j + \alpha}, \quad i = 0, 1, \ldots, n - 1,
\]

    potem za $t_0 = 0$ in $t_1 = 1$ velja:
    \[
    \frac{d^r}{dt^r} p_n(t) \bigg|_{t=t_0} = \frac{d^r}{dt^r} \tilde{p}_{n-1}(t) \bigg|_{t=t_0}, \quad 0 \leq r \leq \alpha - 1.
    \]
\end{frame}

\begin{frame}{Izbira uteži}
\[
\begin{array}{c|c|ccccccc}
\toprule
\textbf{Parameter} & n & \lambda_0 & \lambda_1 & \lambda_2 & \lambda_3 & \lambda_4 & \lambda_5 & \lambda_6 \\
\midrule
\alpha = 1 & 3 & 0 & \frac{1}{2} & 1 & & & & \\
           & 4 & 0 & \frac{3}{14} & \frac{11}{14} & 1 & & & \\
           & 5 & 0 & \frac{1}{12} & \frac{1}{2} & \frac{11}{12} & 1 & & \\
\midrule
\alpha = 2 & 5 & 0 & 0 & \frac{1}{5} & 1 & 1 & & \\
           & 6 & 0 & 0 & \frac{5}{22} & \frac{1}{2} & \frac{17}{22} & 1 & \\
           & 7 & 0 & 0 & \frac{5}{52} & \frac{1}{2} & \frac{47}{52} & 1 & 1 \\
\bottomrule
\end{array}
\]
\end{frame}

\begin{frame}{Posplošitev znižanja stopnje}
    Doslej smo zniževali stopnjo le iz \(n\) na \(n-1\). Sedaj želimo splošnejši algoritem za znižanje stopnje iz \(n\) do \(m\), kjer \(m < n\).
    Postopek je iterativen: na vsakem koraku zmanjšamo stopnjo krivulje za eno, dokler ne dosežemo želene stopnje.
    Ključne lastnosti:
        \begin{itemize}
            \item Nova krivulja in originalna krivulja se ujemata v \(\alpha-1\) prvih odvodih na robnih točkah.
            \item \(L^2\)-norma minimizira razdaljo med krivuljama.
        \end{itemize}
    Uporaba drugih norm, npr. \(d_\infty\), ne bi zagotovila minimizacije z iterativnim algoritmom.
\end{frame}

\begin{frame}{Algoritem}
    \textbf{Vhodni podatki:} \(\{b_0, b_1, \ldots, b_n\}, k, \alpha \)
    \begin{itemize}
        \item \(c_0 = b_0, \quad c_{n-1} = b_n\)
        \item \(\lambda = lambda(n, \alpha)\)
    \end{itemize}

    \textbf{Postopek:}
    \begin{enumerate} % Added enumerate here to encapsulate the process
        \item \textbf{While} \(k > 0\) \textbf{do}
        \begin{enumerate}
            \item \textbf{For} \(i = 1, \ldots, n - 2\) \textbf{do}
            \[
            c_i^{(1)} = \frac{1}{n-i}\left(n b_i - i c_{i-1}^{(1)}\right)
            \]
            \[
            c_{n-i-1}^{(2)} = \frac{1}{n-i+1}\left(n b_{n-i+1} - (i - 1) c_{n-i+1}^{(2)}\right)
            \]
            \item \textbf{For} \(i = 1, \ldots, n - 2\) \textbf{do}
            \[
            c_i = (1 - \lambda_i)c_i^{(1)} + \lambda_i c_i^{(2)}
            \]
            \item \(k = k - 1\)
        \end{enumerate}
    \end{enumerate}
\end{frame}




\begin{frame}
  Primeri v Matlabu.
\end{frame}

\end{document}