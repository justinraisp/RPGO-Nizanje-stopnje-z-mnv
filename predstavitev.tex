\documentclass{beamer}

% Nastavitve stila predstavitve
\usetheme{Madrid}
\usecolortheme{beaver}

% Naslovna stran
\title[Nižanje stopnje Bézierjevih krivulj]{Nižanje stopnje Bézierjevih krivulj z metodo najmanjših kvadratov}
\author{Luka Polanič, Justin Raišp}
\date{Ljubljana, 2024}

\begin{document}

% Naslovni diapozitiv
\begin{frame}
    \titlepage
\end{frame}

% Motivacija
\begin{frame}{Motivacija}
    Podano imamo Bézierjevo krivuljo stopnje $n$:
    \[ p_n(t) = \sum_{i=0}^n b_i B_i^n(t), \quad t \in [0, 1], \]
    kjer so $b_i$ kontrolne točke in $B_i^n(t)$ Bernsteinovi bazni polinomi stopnje $n$.
    Naš cilj je poiskati Bézierjevo krivuljo stopnje $m < n$,
    \[ \tilde{p}_m(t) = \sum_{i=0}^m c_i B_i^m(t), \quad t \in [0, 1], \]
\end{frame}

% Definicije
\begin{frame}{Metoda najmanjših kvadratov}
  Kontrolne točke $c_i$ določimo tako, da minimiziramo $L_2$-normo med krivuljama $p_n$ in $\tilde{p}_m$, pri čemer je $L_2$-norma definirana kot:
  \[ d_2(p_n, \tilde{p}_m) = \sqrt{\int_0^1 \|p_n(t) - \tilde{p}_m(t)\|^2 \, dt}, \]
  kjer je $\|p_n(t) - \tilde{p}_m(t)\|^2$ kvadrat evklidske razdalje med krivuljama.
\end{frame}

% Osnovni izrek
\begin{frame}{Konstrukcija}
Nižanje stopnje iz $n$ na $n-1$. Če bi želeli zvišati stopnjo Bézierjeve krivulje, bi uporabili naslednjo zvezo: 
\[
b_i = \frac{i}{n} c_{i-1} + \frac{n-i}{n} c_i, \quad i = 0, 1, \ldots, n.
\]
Sedaj lahko na dva načina izrazimo zaporedje neznanih kontrolnih točk $\{c_i\}_{i=0}^{n-1}$.
Dobimo dva sistema enačb, pri čemer bomo v obeh primerih zanemarili eno enačbo,
saj bi bil sistem sicer predoločen.
\end{frame}

\begin{frame}
  \begin{itemize}
    \item Zanemarimo zadnjo enačbo za $i = n$ in dobimo
    \[
    c_i^{(I)} = \frac{1}{n - i} \left( n b_i - i c_{i-1}^{(I)} \right) \quad \text{za} \quad i = 0, 1, \ldots, n - 1.
    \]
    Dodatno upoštevamo, da je $c_{-1} = 0$.
    \item Pri drugi izražavi zanemarimo prvo enačbo, tj. za $i = 0$, in dobimo
    \[
    c_{i-1}^{(II)} = \frac{1}{i} \left( n b_i - (n - i) c_i^{(II)} \right) \quad \text{za} \quad i = n, \ldots, 1.
    \]
  \end{itemize}
  Množici kontrolnih točk $\{c_i^{(I)}\}_{i=0}^{n-1}$ in $\{c_i^{(II)}\}_{i=0}^{n-1}$ predstavljata kontrolna poligona za dve različni Bézierjevi krivulji stopnje $n - 1$. Označimo ju z $\tilde{p}_{n-1}^{(I)}$ in $\tilde{p}_{n-1}^{(II)}$.
\end{frame}

\begin{frame}
  Sedaj moramo poiskati še kontrolne točke $\{c_i\}_{i=0}^{n-1}$ končne Bézierjeve krivulje $\tilde{p}_{n-1}$.
  Za kontrolne točke vzamemi linearno kombinacijo točk, določenih z zgornjima izrazoma:
  \[ c_i = (1 - \lambda_i) \cdot c_i^{(I)} + \lambda_i \cdot c_i^{(II)} \quad \text{za} \quad i = 0, 1, \ldots, n - 1. \]
  Z uvedbo faktorjev $\{\lambda_i \in \mathbb{R}\}$ prevedemo problem iskanja kontrolnih točk $\{c_i\}_i$ na problem računanja ustreznih faktorjev $\{\lambda_i\}_i$, za katere velja, da je $d_2(p_n, \tilde{p}_{n-1})$ minimalna.
\end{frame}

\begin{frame}{Izbira uteži}
   Da se pokazati, da če za krivuljo $p_n$ z $\Delta_n b_0 \neq 0$ in $2\alpha \leq n$, izberemo faktorje $\lambda_i$ kot: 
    \[
    \lambda_i = \left( \frac{2n}{n + 2\alpha} \right)^{-1} \cdot \sum_{j=0}^i \left( \frac{n}{j - \alpha} \right) \left( \frac{n}{j + \alpha} \right), \quad i = 0, 1, \ldots, n - 1,
    \]
    potem za $t_0 = 0$ in $t_1 = 1$ velja:
    \[
    \frac{d^r}{dt^r} p_n(t) \bigg|_{t=t_0} = \frac{d^r}{dt^r} \tilde{p}_{n-1}(t) \bigg|_{t=t_0}, \quad 0 \leq r \leq \alpha - 1.
    \]
\end{frame}

% Algoritem
\begin{frame}{Algoritem}
    \begin{itemize}
        \item Začetni podatki: kontrolne točke $b_0, \ldots, b_n$ in ciljna stopnja $m$.
        \item Postopek:
        \begin{enumerate}
            \item Izračun $c_i^{(I)}$ in $c_i^{(II)}$ za stopnjo $n$ do $n-1$.
            \item Izračun kombinacije $c_i$ z utežmi $\lambda_i$.
            \item Ponavljanje postopka za nižanje stopnje do $m$.
        \end{enumerate}
        \item Lastnosti algoritma:
        \begin{itemize}
            \item Ohranja zveznost do reda $\alpha-1$ v robnih točkah.
            \item $L_2$-norma ostaja minimalna.
        \end{itemize}
    \end{itemize}
\end{frame}

\begin{frame}
  Primeri v Matlabu.
\end{frame}

\end{document}
